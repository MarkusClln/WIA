%% Preambel
\documentclass[conference,compsoc,final,a4paper]{IEEEtran}
\usepackage[utf8]{inputenx}

%% Bitte legen Sie hier den Titel und den Autor der Arbeit fest
\newcommand{\autoren}[0]{Cöllen, Markus}
\newcommand{\dokumententitel}[0]{Kann Malware in Android-Apps automatisch gefunden
werden?
(Verschiedene Analysemethoden für Android-Apps)}

\input{preambel} % Weitere Einstellungen aus einer anderen Datei lesen

% Eigentliches Dokument beginnt hier
% ----------------------------------------------------------------------------------------------------------

% Kurze Zusammenfassung des Dokuments
\begin{abstract}
An dieser Stelle steht eine kurze Zusammenfassung des Inhaltes des Dokuments.
\end{abstract}

\tableofcontents

\section{Einleitung}

\section{Grundlagen}

\subsection{Android und Sicherheitslücken}

\subsection{App Store}

\subsection{Malware}

\subsubsection{Android als Ziel}

\subsubsection{Klassifizierung}

\section{Analysemethoden}

\subsection{Statische Analyse}

\subsubsection{Data Flow}

\subsubsection{Control Flow}

\subsection{Dynamische Analyse}

\section{FlowDroid}

\section{Crowdroid}

\section{Fazit}

 Eine Abkürzung = \ac{A2A} 
 \cite[S.1]{Feng:2014:ASD:2635868.2635869} 
 \cite{burguera2011crowdroid}
 \cite{yan2012droidscope}
 \cite{bartsch2014zertifizierte}
 \cite{Ding:2017:MMG:3058060.3058065}
 \cite{Spreitzenbarth2015}
 \cite{Cesare:2010:CMU:1862294.1862301}
 \cite{Arzt:2014:FPC:2666356.2594299}
 \cite{6394931}

%% --------------------------------------------------------------------

\section*{Abkürzungen}
\addcontentsline{toc}{section}{Abkürzungen}

\begin{acronym}
\acro{A2A}{Application-to-Application}
\end{acronym}

% Literaturverzeichnis
\addcontentsline{toc}{section}{Literatur}
\printbibliography

\end{document}
